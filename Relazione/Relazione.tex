\input{preamble.tex}

\title{Ingegneria dei Sistemi Software M}
\date{A.A. 2017--2018}
\author{Marco Boschi, Marco Rossini}

\begin{document}

\maketitletoc

\section{Analisi dei requisiti}
Il robot DDR è un dispositivo in grado di muoversi in un ambiente attraverso comandi remoti per avanzare, indietreggiare o girare di \ang{90} a destra o sinistra oltre al comando di stop. L'ambiente in cui dovrà muoversi è una stanza quadrata che può essere sia fisica che virtuale. La stanza è dotata di due sonar: uno, detto \texttt{sonar1}, è posto in alto a sinistra e rivolto verso il basso, l'altro, detto \texttt{sonar2}, è posto in basso a destra rivolto verso sinistra.

Il robot dovrà muoversi da una posizione iniziale intercettata da \texttt{sonar1}, detta \texttt{start-point}, e muoversi lungo la stanza fino alla posizione finale, detta \texttt{end-point}, intercettata da \texttt{sonar2} mentre pulisce il pavimento, \textit{richiedendo quindi di passare sulla maggior superficie possibile del pavimento}.

Per \textit{sonar} si intende un dispositivo, reale o virtuale, che sfrutta una qualche tecnologia per individuare quando il robot passa davanti a questo.

Il robot è controllato attraverso un'interfaccia grafica, detta \texttt{console}, accessibile ad un utente umano autorizzato dall'inserimento di credenziali riconosciute dal sistema. L'attività di pulizia è innescata dall'invio del comando \texttt{START} da parte dell'utente autorizzato che controlla il robot. Ciò avviene mediante un qualunque dispositivo in grado di connettersi al robot.

Il robot lavora solo se la temperatura dell'ambiente non è superiore ad una soglia prefissata e se l'ora corrente è all'interno di un intervallo prefissato. Il robot è dotato di un sensore di temperatura e di un orologio per verificare queste condizioni.

Durante l'attività di pulizia, il robot deve far lampeggiare una lampada a led HUE, ovvero una lampadina, connessa alla rete, che può essere controllata in luminosità, accensione, spegnimento e colore attraverso API RESTful.

Durante l'attività di pulizia, il robot deve inoltre aver cura di evitare ostacoli fissi presenti nella stanza.

L'attività di pulizia deve terminare se:
\begin{itemize}
\item l'utente autorizzato invia il comando \texttt{STOP} dalla console;
\item la temperatura dell'ambiente supera la soglia prefissata;
\item l'ora corrente esce dall'intervallo prefissato;
\item il robot non riesce, in alcun modo, ad evitare un certo ostacolo;
\item il robot ha terminato l'attività di pulizia, ovvero raggiunge la posizione finale.
\end{itemize}

\section{Analisi del problema e scelte progettuali}
Analizzando i progetti realizzati e presentati a lezione, si ha già a disposizione la console, realizzata come un frontend server in tecnologia Node.js. Essa presenta l'autenticazione degli utenti attraverso una coppia username-password come credenziali di accesso ed i comandi per pilotare il robot nelle sue azioni di base. Possono essere aggiunti facilmente i pulsanti per inviare i comandi di \texttt{START} e \texttt{STOP}.

Il sistema che si deve realizzare è distribuito, pertanto è necessario individuare un sistema di comunicazione tra le parti, ossia il frontend, il robot, i sensori e i sonar. Ciò può essere individuato nello scambio di eventi supportato dall'architettura MQTT già integrata all'interno della console che si ha a disposizione. 

Nella scelta di lavorare all'interno di un ambiente virtuale, si ha a disposizione il progetto ConfigurableThreejsApp, che offre l'ambiente virtuale, dotato dei sonar, ed il robot virtuale. Esso è accessibile attraverso un'interfaccia web e controllabile attraverso una socket TCP.

\section{Log}

\subsection{1° settimana}
Ci siamo concentrati sull'analisi dei requisiti e del problema a livello generale per individuare le parti che abbiamo già a disposizione da progetti svolti a lezione. Queste parti sono state assemblate assieme a un mock per i dispositivi hardware (sensori e attuatori) richiesti per avere un primo prototipo funzionante, ma senza logica applicativa, del sistema.

\end{document}
