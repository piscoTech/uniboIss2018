\input{preamble.tex}

\title{Ingegneria dei Sistemi Software M}
\date{A.A. 2017--2018}
\author{Marco Boschi, Marco Rossini}

\begin{document}

\maketitletoc

\section{Analisi dei requisiti}
Il robot DDR è un dispositivo in grado di muoversi in un ambiente attraverso comandi remoti per avanzare, indietreggiare o girare di \ang{90} a destra o sinistra oltre al comando di stop. L'ambiente in cui dovrà muoversi è una stanza quadrata che può essere sia fisica che virtuale. La stanza è dotata di due sonar: uno, detto \texttt{sonar1}, è posto in alto a sinistra e rivolto verso il basso, l'altro, detto \texttt{sonar2}, è posto in basso a destra rivolto verso sinistra.

Il robot dovrà muoversi da una posizione iniziale intercettata da \texttt{sonar1}, detta \texttt{start-point}, e muoversi lungo la stanza fino alla posizione finale, detta \texttt{end-point}, intercettata da \texttt{sonar2} mentre pulisce il pavimento, \textit{richiedendo quindi di passare sulla maggior superficie possibile del pavimento}.

Per \textit{sonar} si intende un dispositivo, reale o virtuale, che sfrutta una qualche tecnologia per individuare quando il robot passa davanti a questo.

\section{Analisi del problema e scelte progettuali}

\section{Log}

\subsection{1° settimana}
Ci siamo concentrati sull'analisi dei requisiti e del problema a livello generale per individuare le parti che abbiamo già a disposizione da progetti svolti a lezione. Queste parti sono state assemblate assieme a un mock per i dispositivi hardware (sensori e attuatori) richiesti per avere un primo prototipo funzionante, ma senza logica applicativa, del sistema.

\end{document}
